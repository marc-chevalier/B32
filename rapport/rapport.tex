\input{src/macros.tex}

\title{Cryptanalyse linéaire}
\author{
    William \textsc{Aufort}\\
    Marc \textsc{Chevalier}
}
\date{\today}

\begin{document}
\maketitle

\section*{Question 2}

Notre programme nous donne la matrice $L$ suivante :

\setcounter{MaxMatrixCols}{16}
$ L = \quad \begin{pmatrix}
16 & 8 & 8 & 8 & 8 & 8 & 8 & 8 & 8 & 8 & 8 & 8 & 8 & 8 & 8 & 8 \\
8 & 10 & 8 & 6 & 8 & 14 & 8 & 10 & 10 & 8 & 6 & 8 & 6 & 8 & 10 & 8 \\
8 & 4 & 8 & 8 & 6 & 10 & 6 & 6 & 8 & 8 & 4 & 8 & 10 & 10 & 6 & 10 \\
8 & 6 & 8 & 6 & 6 & 8 & 6 & 8 & 10 & 8 & 10 & 8 & 8 & 10 & 8 & 2 \\
8 & 10 & 8 & 10 & 8 & 6 & 8 & 6 & 14 & 8 & 6 & 8 & 10 & 8 & 10 & 8 \\
8 & 8 & 4 & 8 & 8 & 8 & 12 & 8 & 8 & 12 & 8 & 8 & 8 & 12 & 8 & 8 \\
8 & 10 & 4 & 10 & 10 & 8 & 6 & 8 & 6 & 4 & 6 & 8 & 8 & 10 & 8 & 6 \\
8 & 8 & 8 & 8 & 10 & 10 & 10 & 2 & 8 & 8 & 8 & 8 & 6 & 6 & 6 & 6 \\
8 & 4 & 6 & 6 & 10 & 6 & 8 & 8 & 8 & 8 & 6 & 10 & 6 & 6 & 12 & 8 \\
8 & 10 & 6 & 8 & 2 & 8 & 8 & 6 & 6 & 8 & 8 & 10 & 8 & 6 & 10 & 8 \\
8 & 8 & 10 & 10 & 8 & 8 & 10 & 10 & 8 & 8 & 6 & 14 & 8 & 8 & 6 & 6 \\
8 & 6 & 10 & 12 & 8 & 10 & 10 & 8 & 6 & 8 & 8 & 6 & 10 & 8 & 12 & 6 \\
8 & 6 & 6 & 8 & 6 & 8 & 12 & 10 & 10 & 4 & 8 & 6 & 8 & 6 & 6 & 8 \\
8 & 8 & 10 & 10 & 6 & 6 & 8 & 8 & 8 & 8 & 6 & 6 & 2 & 10 & 8 & 8 \\
8 & 6 & 6 & 12 & 8 & 10 & 6 & 8 & 10 & 8 & 12 & 10 & 6 & 8 & 8 & 10 \\
8 & 8 & 10 & 6 & 8 & 8 & 10 & 6 & 8 & 4 & 10 & 10 & 8 & 12 & 10 & 10

\end{pmatrix}$

\section*{Question 3}

Les probabilités les plus éloignées de 0.5 (hormis 0) sont 0.125 et 0.875.
Elles sont atteintes respectivement pour les couples :
\begin{enumerate}
\item $p_{3,15} = p_{7,7} = p_{9,4} = p_{13,12} = 0.125$
\item $p_{1,5} = p_{4,8} = p_{10,11} = 0.875$
\end{enumerate}

Sachant que l'on cherche à approximer la fonction de chiffrement par la fonction linéaire, on souhaite avoir une probabilité de ressemblance importantes, les couples $(a,b)$ tels que $p_{a,b} = 0.875$ sont donc les plus utiles pour la cryptanalyse linéaire.

\section*{Question 4}

Avec les notations introduites, on a $A.m = a.m^{(0)}$ et $P(B).x_1 = P(B).F(x_0,K_1) = P(B).(P(S(x_0)\oplus K_1)$.

Soit $K_3$ telle que $K_1 = P(K_3)$. On a :

$P(B).x_1 = P(B).(P(S(x_0))\oplus P(K_3)) = P(B).P(S(x_0) \oplus K_3) = B.(S(x_0) \oplus K_3) = b.(S(x_0^{(0)}) \oplus K_3^{(0)})$.

D'où $\prob{A.m = P(B).x_1} = \prob{a.m^{(0)}= b.(S(x_0^{(0)}) \oplus K_3^{(0)})}$.

Avec $x_0^{(0)} = m^{(0)} \oplus K_0^{(0)}$.

%TODO : En gros l'idée est que l'ajout de clé ne change pas la distribution uniforme des messages. A justifier !

\end{document}

